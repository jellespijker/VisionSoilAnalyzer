The software for the VSA is written in C++ on a Desktop computer running Linux 3.19.0-18-generic \#. \textbf{Ubuntu 15.04} is the environment of choice. The prefered IDE is \textbf{QT Creator Community edition}. This is an opensource IDE and available for Linux/Windows/Mac. Version control is done via \textbf{Github} the main project page is VisionSoilAnalyzer - project page (\url{http://peer23peer.github.io/VisionSoilAnalyzer/}). The following page. The packages and installation steps depicted below are the absolute minimum needed to start development. It is advised to make a clone of your harddisk after the development environment setup is complete. Clonezilla is the recommended tool for this.

Below is a list of used libraries during and their installation instructions:

\paragraph{Common packages}
\begin{sBox}
	sudo apt-get install build-essential cmake git libgtk2.0-dev pkg-config libavcodec-dev libavformat-dev libswscale-dev python-dev python-numpy libtbb2 libtbb-dev libjpeg-dev libpng-dev libtiff-dev libjasper-dev libdc1394-22-dev libv4l-dev v4l-utils libqt5multimediawidgets5 clang libboost-all-dev cheese cmake-qt-gui qt-sdk libgstreamer0.10-dev libgstreamer-plugins-base0.10-dev libv4l-dev libtbb-dev libqt4-dev libfaac-dev libmp3lame-dev libopencore-amrnb-dev libopencore-amrwb-dev libtheora-dev libvorbis-dev libxvidcore-dev x264 v4l-utils unzip libopencv-dev build-essential cmake git libgtk2.0-dev pkg-config python-dev python-numpy libdc1394-22 libdc1394-22-dev libjpeg-dev libpng12-dev libjasper-dev libavcodec-dev libavformat-dev libswscale-dev libxine2-dev libtiff5-dev libgstreamer0.10-dev libpython3-all-dev libpython-all-dev libbz2-dev valgrind python3-numpy
\end{sBox}

\paragraph{Nvidia driver (820M)}
\begin{sBox}
	sudo apt-get purge nvidia*\\
	sudo add-apt-repository ppa:graphics-drivers/ppa\\
	sudo apt-get update\\
	sudo apt-get install nvidia-355 nvidia-settings\\
	sudo nvidia-xconfig\\
	sudo apt-get install bumblebee bbswitch-dkms primus\\
	sudo systemctl enable bumblebeed\\
	sudo echo "i915" >> /etc/modules-load.d/modules.conf \&\& sudo echo "bbswitch" >> /etc/modules-load.d/modules.conf\\  
	sudo ln -s /usr/lib/nvidia-current /usr/lib/nvidia-355\\
	sudo ln -s /usr/lib32/nvidia-current /usr/lib32/nvidia-355\\
	sudo nano /etc/bumblebee/bumblebee.conf\\
	change the parameter `Driver=` > `Driver=nvidia\\
	change the parameter `KernelDriver=nvidia-current` > `KernelDriver=nvidia-355\\
	restart the computer\\
\end{sBox}


\paragraph{CUDA}
\begin{sBox}
	wget http://developer.download.nvidia.com/compute/cuda/7\_0/Prod/local\_installers/rpmdeb/cuda-repo-ubuntu1410-7-0-local\_7.0-28\_amd64.deb\\
	sudo dpkg -i cuda-repo-ubuntu1410-7-0-local\_7.0-28\_amd64.deb\\
	sudo apt-get update\\
	sudo apt-get install cuda\\
	export PATH=/usr/local/cuda-7.0/bin:\$PATH\\
	export LD\_LIBRARY\_PATH=/usr/local/cuda-7.0/lib64:\$LD\_LIBRARY\_PATH\
\end{sBox}  
 
\paragraph{Qt and Qt Creator}
\begin{sBox}
	wget http://download.qt.io/official\_releases/online\_installers/qt-unified-linux-x64-online.run\\
	sudo chmod +x qt-unified-linux-x64-online.run\\
	./qt-unified-linux-x64-online.run\\
\end{sBox}
 
\paragraph{OpenCV 3.0 beta}
\begin{sBox}
	cd \~\\   
	git clone https://github.com/Itseez/opencv.git\\
	cd opencv\\
	mkdir release\\
	cd release\\
	Enter the following command for an NVIDIA CUDA enabled environment:\\
	cmake -D CMAKE\_BUILD\_TYPE=RELEASE -D CMAKE\_INSTALL\_PREFIX=/usr/local -D BUILD\_CUDA\_STUBS=ON -D BUILD\_DOCS=OFF -D BUILD\_JPEG=ON -D BUILD\_PNG=ON -D BUILD\_TESTS=OFF -D BUILD\_WITH\_DEBUG\_INFO=OFF -D CUDA\_FAST\_MATH=ON -D ENABLE\_FAST\_MATH=ON -D WITH\_CUBLAS=ON WITH\_OPENGL=ON WITH\_QT=ON ..\\
	Or the command below for a computer that has no NVIDIA CUDA capabilities:\\  
	cmake -D CMAKE\_BUILD\_TYPE=RELEASE -D CMAKE\_INSTALL\_PREFIX=/usr/local -D WITH\_CUDA=OFF -D BUILD\_DOCS=OFF -D BUILD\_JPEG=ON -D BUILD\_PNG=ON -D BUILD\_TESTS=OFF -D BUILD\_WITH\_DEBUG\_INFO=OFF -D ENABLE\_FAST\_MATH=ON -D WITH\_OPENGL=ON -D WITH\_QT=ON ..\\
	make -jnumber of processors\\
	sudo make install\\
	sudo /bin/bash -c 'echo "/usr/local/lib" > /etc/ld.so.conf.d/opencv.conf'\\
	sudo ldconfig\\
\end{sBox}  
  
\paragraph{ZLib}
\begin{sBox}
	wget http://zlib.net/zlib-1.2.8.tar.gz\\
	tar xf zlib-1.2.8.tar.gz\\
	cd zlib-1.2.8\\
	./configure\\
	make -jnumber of processors\\
	sudo make install\\
\end{sBox}

\paragraph{Testing and benchmarking}
Testing is done using the \textbf{QT unit test framework} results are verified against know results. Which are calculated via \textbf{Matlab},\textbf{Mathematica} or \textbf{Python}. Benchmarks are done using the QT unit test framework and will test multiple solutions. Solutions that are deemed obsolete by the benchmark results will not be removed but be renamed with a \_ in front of the function name \_FunctionName. \textbf{Valgrind} is used to determine memory leakages and function profiles. These profiles will be the guide which determine the priority of functions to be optimized.

\paragraph{Source documentation}
The detailed Doxygen documentation of the current program and libs can be found [source documentation](\url{http://peer23peer.github.io/VisionSoilAnalyzer/Doxygen/html/index.html}). It consists of detailed graphs, the complete source code and the internal structure of the program. This page is generated using \textbf{Doxygen}. This programs grabs source code comments and converts them to documentation. 
