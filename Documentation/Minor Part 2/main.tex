%%%%%%%%%%%%%%%%%%%%%%%%%%%%%%%%%%%%%%%%%
% The Legrand Orange Book
% LaTeX Template
% Version 2.0 (9/2/15)
%
% This template has been downloaded from:
% http://www.LaTeXTemplates.com
%
% Mathias Legrand (legrand.mathias@gmail.com) with modifications by:
% Vel (vel@latextemplates.com)
%
% License:
% CC BY-NC-SA 3.0 (http://creativecommons.org/licenses/by-nc-sa/3.0/)
%
% Compiling this template:
% This template uses biber for its bibliography and makeindex for its index.
% When you first open the template, compile it from the command line with the 
% commands below to make sure your LaTeX distribution is configured correctly:
%
% 1) pdflatex main
% 2) makeindex main.idx -s StyleInd.ist
% 3) biber main
% 4) pdflatex main x 2
%
% After this, when you wish to update the bibliography/index use the appropriate
% command above and make sure to compile with pdflatex several times 
% afterwards to propagate your changes to the document.
%
% This template also uses a number of packages which may need to be
% updated to the newest versions for the template to compile. It is strongly
% recommended you update your LaTeX distribution if you have any
% compilation errors.
%
% Important note:
% Chapter heading images should have a 2:1 width:height ratio,
% e.g. 920px width and 460px height.
%
%%%%%%%%%%%%%%%%%%%%%%%%%%%%%%%%%%%%%%%%%

%----------------------------------------------------------------------------------------
%	PACKAGES AND OTHER DOCUMENT CONFIGURATIONS
%----------------------------------------------------------------------------------------

\documentclass[11pt,fleqn]{book} % Default font size and left-justified equations

%----------------------------------------------------------------------------------------

\input{structure} % Insert the commands.tex file which contains the majority of the structure behind the template

\begin{document}

%----------------------------------------------------------------------------------------
%	TITLE PAGE
%----------------------------------------------------------------------------------------

\begingroup
\thispagestyle{empty}
\begin{tikzpicture}[remember picture,overlay]
\coordinate [below=15cm] (midpoint) at (current page.north);
\node at (current page.north west)
{\begin{tikzpicture}[remember picture,overlay]
\node[anchor=north west,inner sep=0pt] at (0,0) {\includegraphics[width=\paperwidth]{background.jpg}}; % Background image
\draw[anchor=north] (midpoint) node [fill=ocre!30!white,fill opacity=0.6,text opacity=1,inner sep=1cm]{\Huge\centering\bfseries\sffamily\parbox[c][][t]{\paperwidth}{\centering Vision Soil Analyzer\\[15pt] % Book title
{\Large Product design of a vision based soil analyzer}\\[20pt] % Subtitle
{\huge Jelle Spijker}}}; % Author name
\end{tikzpicture}};
\end{tikzpicture}
\vfill
\endgroup

%----------------------------------------------------------------------------------------
%	COPYRIGHT PAGE
%----------------------------------------------------------------------------------------

\newpage
~\vfill
\thispagestyle{empty}

\noindent Copyright \copyright\ 2015 Jelle Spijker\\ % Copyright notice

\noindent \textsc{Published by Royal IHC}\\ % Publisher

\noindent \textsc{book-website.com}\\ % URL

\noindent This document remains the property of “IHC Holland B.V.” All rights
reserved. This document or any part thereof may not be made public
or disclosed, copied or otherwise reproduced or used in any form or
by any means, without prior permission in writing from “IHC Holland
B.V.” \\ % License information

\noindent \textit{First printing, September 2015} % Printing/edition date

%----------------------------------------------------------------------------------------
%	TABLE OF CONTENTS
%----------------------------------------------------------------------------------------

\chapterimage{rocky-soil.jpg} % Table of contents heading image

\pagestyle{empty} % No headers

\tableofcontents % Print the table of contents itself

\cleardoublepage % Forces the first chapter to start on an odd page so it's on the right

\pagestyle{fancy} % Print headers again

%----------------------------------------------------------------------------------------
%	PART
%----------------------------------------------------------------------------------------
\newpage
\part{Part One}

%----------------------------------------------------------------------------------------
%	Introduction
%----------------------------------------------------------------------------------------
\chapter{Introduction}
\lipsum[3-5]

%----------------------------------------------------------------------------------------
%	CHAPTER 1
%----------------------------------------------------------------------------------------

\chapterimage{rocky-soil.jpg} % Chapter heading image

\chapter{Functional Design}

\section{Global Input-Proces-Output}\index{Global Input-Proces-Output}

\lipsum[1-7] % Dummy text

\section{Specifications}\index{Citation}

This statement requires citation \cite{book_key}; this one is more specific \cite[122]{article_key}.
\subsection{Functional requirements}
\begin{description}
	\item[Name] Description
	\item[Word] Definition
	\item[Comment] Elaboration
\end{description}
\subsection{Technical requirements}
\begin{description}
\item[Name] Description
\item[Word] Definition
\item[Comment] Elaboration
\end{description}

%----------------------------------------------------------------------------------------
%	CHAPTER 2
%----------------------------------------------------------------------------------------

\chapter{Technical Design}

\section{Theorems}\index{Theorems}

This is an example of theorems.

\section{Several equations}\index{Theorems!Several Equations}
This is a theorem consisting of several equations.

\begin{theorem}[Name of the theorem]
In $E=\mathbb{R}^n$ all norms are equivalent. It has the properties:
\begin{align}
& \big| ||\mathbf{x}|| - ||\mathbf{y}|| \big|\leq || \mathbf{x}- \mathbf{y}||\\
&  ||\sum_{i=1}^n\mathbf{x}_i||\leq \sum_{i=1}^n||\mathbf{x}_i||\quad\text{where $n$ is a finite integer}
\end{align}
\end{theorem}

\section{Single Line}\index{Theorems!Single Line}
This is a theorem consisting of just one line.

\begin{theorem}
A set $\mathcal{D}(G)$ in dense in $L^2(G)$, $|\cdot|_0$. 
\end{theorem}

\chapter{Vision design}

\section{Definitions}\index{Definitions}

This is an example of a definition. A definition could be mathematical or it could define a concept.

\begin{definition}[Definition name]
Given a vector space $E$, a norm on $E$ is an application, denoted $||\cdot||$, $E$ in $\mathbb{R}^+=[0,+\infty[$ such that:
\begin{align}
& ||\mathbf{x}||=0\ \Rightarrow\ \mathbf{x}=\mathbf{0}\\
& ||\lambda \mathbf{x}||=|\lambda|\cdot ||\mathbf{x}||\\
& ||\mathbf{x}+\mathbf{y}||\leq ||\mathbf{x}||+||\mathbf{y}||
\end{align}
\end{definition}

%------------------------------------------------

\section{Notations}\index{Notations}

\begin{notation}
Given an open subset $G$ of $\mathbb{R}^n$, the set of functions $\varphi$ are:
\begin{enumerate}
\item Bounded support $G$;
\item Infinitely differentiable;
\end{enumerate}
a vector space is denoted by $\mathcal{D}(G)$. 
\end{notation}

%------------------------------------------------
\part{Two}
\chapter{Technical Realisation}

\subsection{Electrical design}

\subsection{Design}

\chapter{Vision realisation}\index{Image Acquisition}

\section{Image Acquisition}\index{Image Acquisition}

This is an example of a remark.

\begin{remark}
The concepts presented here are now in conventional employment in mathematics. Vector spaces are taken over the field $\mathbb{K}=\mathbb{R}$, however, established properties are easily extended to $\mathbb{K}=\mathbb{C}$.
\end{remark}

%------------------------------------------------

\section{Corollaries}\index{Corollaries}

This is an example of a corollary.

\begin{corollary}[Corollary name]
The concepts presented here are now in conventional employment in mathematics. Vector spaces are taken over the field $\mathbb{K}=\mathbb{R}$, however, established properties are easily extended to $\mathbb{K}=\mathbb{C}$.
\end{corollary}

%------------------------------------------------

\section{Propositions}\index{Propositions}

This is an example of propositions.

\subsection{Several equations}\index{Propositions!Several Equations}

\begin{proposition}[Proposition name]
It has the properties:
\begin{align}
& \big| ||\mathbf{x}|| - ||\mathbf{y}|| \big|\leq || \mathbf{x}- \mathbf{y}||\\
&  ||\sum_{i=1}^n\mathbf{x}_i||\leq \sum_{i=1}^n||\mathbf{x}_i||\quad\text{where $n$ is a finite integer}
\end{align}
\end{proposition}

\subsection{Single Line}\index{Propositions!Single Line}

\begin{proposition} 
Let $f,g\in L^2(G)$; if $\forall \varphi\in\mathcal{D}(G)$, $(f,\varphi)_0=(g,\varphi)_0$ then $f = g$. 
\end{proposition}

%------------------------------------------------

\section{Examples}\index{Examples}

This is an example of examples.

\subsection{Equation and Text}\index{Examples!Equation and Text}

\begin{example}
Let $G=\{x\in\mathbb{R}^2:|x|<3\}$ and denoted by: $x^0=(1,1)$; consider the function:
\begin{equation}
f(x)=\left\{\begin{aligned} & \mathrm{e}^{|x|} & & \text{si $|x-x^0|\leq 1/2$}\\
& 0 & & \text{si $|x-x^0|> 1/2$}\end{aligned}\right.
\end{equation}
The function $f$ has bounded support, we can take $A=\{x\in\mathbb{R}^2:|x-x^0|\leq 1/2+\epsilon\}$ for all $\epsilon\in\intoo{0}{5/2-\sqrt{2}}$.
\end{example}

\subsection{Paragraph of Text}\index{Examples!Paragraph of Text}

\begin{example}[Example name]
\lipsum[2]
\end{example}

%------------------------------------------------

\section{Exercises}\index{Exercises}

This is an example of an exercise.

\begin{exercise}
This is a good place to ask a question to test learning progress or further cement ideas into students' minds.
\end{exercise}

%------------------------------------------------

\section{Problems}\index{Problems}

\begin{problem}
What is the average airspeed velocity of an unladen swallow?
\end{problem}

%------------------------------------------------

\section{Vocabulary}\index{Vocabulary}

Define a word to improve a students' vocabulary.

\begin{vocabulary}[Word]
Definition of word.
\end{vocabulary}

%----------------------------------------------------------------------------------------
%	PART
%----------------------------------------------------------------------------------------

\part{Part Two}

%----------------------------------------------------------------------------------------
%	CHAPTER 3
%----------------------------------------------------------------------------------------

\chapterimage{chapter_head_1.pdf} % Chapter heading image

\chapter{Presenting Information}

\section{Table}\index{Table}

\begin{table}[h]
\centering
\begin{tabular}{l l l}
\toprule
\textbf{Treatments} & \textbf{Response 1} & \textbf{Response 2}\\
\midrule
Treatment 1 & 0.0003262 & 0.562 \\
Treatment 2 & 0.0015681 & 0.910 \\
Treatment 3 & 0.0009271 & 0.296 \\
\bottomrule
\end{tabular}
\caption{Table caption}
\end{table}

%------------------------------------------------

\section{Figure}\index{Figure}

\begin{figure}[h]
\centering\includegraphics[scale=0.5]{placeholder}
\caption{Figure caption}
\end{figure}

\part[part]{Part Three}
%----------------------------------------------------------------------------------------
%	BIBLIOGRAPHY
%----------------------------------------------------------------------------------------

\chapter*{Bibliography}
\addcontentsline{toc}{chapter}{\textcolor{ocre}{Bibliography}}
\section*{Books}
\addcontentsline{toc}{section}{Books}
\printbibliography[heading=bibempty,type=book]
\section*{Articles}
\addcontentsline{toc}{section}{Articles}
\printbibliography[heading=bibempty,type=article]

%----------------------------------------------------------------------------------------
%	INDEX
%----------------------------------------------------------------------------------------

\cleardoublepage
\phantomsection
\setlength{\columnsep}{0.75cm}
\addcontentsline{toc}{chapter}{\textcolor{ocre}{Index}}
\printindex

%----------------------------------------------------------------------------------------
%	Appendix
%----------------------------------------------------------------------------------------

\newpage
\onecolumn
\appendix
\chapter{SoilMath Library}
\subsection{Genetic Algorithm Class}
\lstinputlisting[language=C++]{../../src/SoilMath/GA.h}
\newpage
\lstinputlisting[language=C++]{../../src/SoilMath/GA.cpp}
\newpage
\subsection{Fast Fourier Transform Class}
\lstinputlisting[language=C++]{../../src/SoilMath/FFT.h}
\newpage
\lstinputlisting[language=C++]{../../src/SoilMath/FFT.cpp}
\newpage
\subsection{Neural Network Class}
\lstinputlisting[language=C++]{../../src/SoilMath/NN.h}
\newpage
\lstinputlisting[language=C++]{../../src/SoilMath/NN.cpp}
\newpage
\subsection{Statistical Class}
\lstinputlisting[language=C++]{../../src/SoilMath/Stats.h}
\newpage
\lstinputlisting[language=C++]{../../src/SoilMath/psd.h}
\newpage
\subsection{General project file}
\lstinputlisting[language=C++]{../../src/SoilMath/SoilMath.h}
\newpage
\lstinputlisting[language=C++]{../../src/SoilMath/CommonOperations.h}
\newpage
\lstinputlisting[language=C++]{../../src/SoilMath/SoilMathTypes.h}
\newpage
\lstinputlisting[language=C++]{../../src/SoilMath/Mat_archive.h}
\newpage
\lstinputlisting[language=C++]{../../src/SoilMath/predict_t_archive.h}
\newpage
\lstinputlisting[language=C++]{../../src/SoilMath/MathException.h}
\newpage
\lstinputlisting[language=C++]{../../src/SoilMath/Sort.h}

\chapter{Hardware Library}
\subsection{Microscope Class}
\lstinputlisting[language=C++]{../../src/SoilHardware/Microscope.h}
\newpage
\lstinputlisting[language=C++]{../../src/SoilHardware/Microscope.cpp}
\newpage
\subsection{Beaglebone Black Class}
\lstinputlisting[language=C++]{../../src/SoilHardware/BBB.h}
\newpage
\lstinputlisting[language=C++]{../../src/SoilHardware/BBB.cpp}
\newpage
\subsection{GPIO Class}
\lstinputlisting[language=C++]{../../src/SoilHardware/GPIO.h}
\newpage
\lstinputlisting[language=C++]{../../src/SoilHardware/GPIO.cpp}
\newpage
\subsection{PWM Class}
\lstinputlisting[language=C++]{../../src/SoilHardware/PWM.h}
\newpage
\lstinputlisting[language=C++]{../../src/SoilHardware/PWM.cpp}
\newpage
\subsection{General project file}
\lstinputlisting[language=C++]{../../src/SoilHardware/Hardware.h}
\newpage
\lstinputlisting[language=C++]{../../src/SoilHardware/ValueOutOfBoundsException.h}
\newpage
\lstinputlisting[language=C++]{../../src/SoilHardware/ADCReadException.h}
\newpage
\lstinputlisting[language=C++]{../../src/SoilHardware/FailedToCreateGPIOPollingThreadException.h}
\newpage
\lstinputlisting[language=C++]{../../src/SoilHardware/FailedToCreateThreadException.h}
\newpage
\lstinputlisting[language=C++]{../../src/SoilHardware/MicroscopeNotFoundException.h}
\newpage
\lstinputlisting[language=C++]{../../src/SoilHardware/CouldNotGrabImageException.h}

\end{document}