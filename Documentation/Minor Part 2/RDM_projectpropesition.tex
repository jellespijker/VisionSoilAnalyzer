Subject: RDM Student project\\
Author: Jelle Spijker\\

\paragraph{Introduction}
This project finds its roots in the minor Embedded Vision Design (EVD) taught at the university of applied sciences HAN. During this minor a portable embedded device was developed which analyses soil samples using a microscope. This Vision Soil Analyser hereafter referred to as VSA, analyses soil samples using the optical properties. It’s main function is: Presenting quantifiable information to a user on the properties of soil: such as colour, texture and structure.

The VSA takes a snapshot from a soil sample, which is placed under a microscope in an closed environment. This digital image is analysed using a multitude of computer vision algorithms. Statistical data is presented to the user in the form a Particle Size Distribution (PSD) and a histogram of the shape classification. The PSD is obtained by calculating the number of pixels for each individual particle, whilst shape classification is determined by describing the contour of each individual particle as mathematical function which undergoes a transformation to the frequency domain. This complex vector then serves as input for an Artificial Neural Network (ANN) where the output classifies each particle in a certain category.

The prototype developed during the minor EVD will serve as a basis for a graduation project of that same student, which initialized the project. This is done for his main course mechanical engineering at the HAN. This graduation project is done under the auspices of MTI Holland. The goal during this second stage is to develop a field ready prototype. In conjunction with the necessary documentation (Technical Dossier). 
Due to the scale of the project, several key problems are identified and separated from the main project. These problems can be tackled by separated student groups.

\paragraph{Problem description}
Due to the transformation from 3D particles to a discrete 2D image certain data is lost. This degradation of data introduces errors in the statistical data. One of the forms of degradations is the overlap of bigger particle onto smaller particles. These particles are identified as an particle with at least the size and the contour of the biggest particles. Thus giving false negatives for the smaller particles and often false positives for the bigger particle.

A solution that will be explored during this stage is the execution of multiple analysis of the same discrete particle population. This will result in an accurate statistical representation of the soil sample placed under the microscope.

The project that the RDM students can tackle can be described as follow:
\begin{sBox}
	Design and build a prototype with which the placement of particles, relative to each other and ranging in sizes from 0.02 - 2 [mm] are randomly changed in a time span of 1 [sec], which is tightly integrated with the main prototype. 
\end{sBox}

The prototype is to be CE compliant and should be build according to technical specifications. It should be described in a Technical Dossier, containing all necessary documents such as: technical drawings (according to mono system), bill of materials, calculation, analysis and design reports.
