With the global Input-Process-Output in mind the functional specifications can be written. This is done by determining all the requirements. These are specification that define a product. It is important to note that there are two types of requirements; Functional and  technical requirements. Each requirement can either be constant or a variable. The constant requirements are the baseline. If the product doesn't fulfills these, it can't be called a soil analyzer. Whilst variable requirement determine how well a product performs. 

\paragraph{Functional Requirements}
Functional requirements describe the functionality and are described in such a way that they are either true or false.

\begin{figure}
	\begin{tabular}{p{1cm}| p{10cm} p{1.5cm}}
		\rowcolor{ocre}
		\hline \textbf{ID} & \textbf{Description} & \textbf{Type} \\ 
		\rowcolor{bananamania}
		\hline \textbf{F1}\label{F1} & \textbf{Quantify color} &  \\ 
		\hline \textbf{F1.1}\label{F1.1} & Determine color in the RGB color model from all humanly visually discernible particles & Const. \\ 
		\hline \textbf{F1.2}\label{F1.2} & Chromatic a* values lie within $3 \sigma$ & Const. \\ 
		\hline \textbf{F1.3}\label{F1.3} & Chromatic b* values lie within $3 \sigma$ & Const. \\ 
		\rowcolor{bananamania}
		\hline \textbf{F2}\label{F2} & \textbf{Quantify texture} &  \\ 
		\hline \textbf{F2.1}\label{F2.1} & The result of an analyzed sample should fall within a probability of at least $P = 0.95$ \% when compared against the result of the same sample, but obtained using the established sieve method. These results are to be compared by Welch's t-test  &  Const. \\ 
		\hline \textbf{F2.2} & No more then 2\% of a extracted blobs may be connected particles & Const. \\ 
		\hline  &  &  \\ 
		\hline  &  &  \\ 
		\hline  &  &  \\ 
		\hline  &  &  \\ 
		\hline  &  &  \\ 
		\hline  &  &  \\ 
		\hline  &  &  \\ 
		\hline  &  &  \\ 
		\hline  &  &  \\ 
		\hline  &  &  \\ 
		\hline  &  &  \\ 
		\hline  &  &  \\ 
		\hline  &  &  \\ 
		\hline  &  &  \\ 
		\hline  &  &  \\ 
		\hline  &  &  \\ 
		\hline  &  &  \\ 
		\hline  &  &  \\ 
		\hline  &  &  \\ 
		\hline  &  &  \\ 
		\hline  &  &  \\ 
		\hline  &  &  \\ 
		\hline 
	\end{tabular} 
	\caption{Functional Requirements}\label{Functional Requirements}
\end{figure}