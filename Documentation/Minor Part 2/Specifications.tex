With the global Input-Process-Output in mind the functional specifications can be written. This is done by determining all the requirements. These are specification that define a product. It is important to note that there are two types of requirements; Functional and  technical requirements. Each requirement can either be constant or a variable. The constant requirements are the baseline. If the product doesn't fulfills these, it can't be called a soil analyzer. Whilst variable requirement determine how well a product performs. 

\paragraph{Functional Requirements}
Functional requirements describe the functionality and are described in such a way that they are either true or false.

\begin{figure}
	\begin{tabular}{|c|c|c|}
		\hline ID & Description & Type \\ 
		\hline F1\label{F1} & Quantify color &  \\ 
		\hline F1.1\label{F1.1} & Determine color from all (with a eye discernible particles) & Const. \\ 
		\hline F1.2\label{F1.2} & Chromatic a* values lie within $3 \sigma$ & Const. \\ 
		\hline F1.3\label{F1.3} & Chromatic b* values lie within $3 \sigma$ & Const. \\ 
		\hline F2\label{F2} & Quantify texture &  \\ 
		\hline F2.1\label{F2.1} & The probability a visually determined PSD belongs to a sieved PSD of the same sample should be at least 95 \% \(Welch's student T-test\) &  Const. \\ 
		\hline F2.2 & No more then 2\% of a extracted blobs may be connected particles & Const. \\ 
		\hline  &  &  \\ 
		\hline  &  &  \\ 
		\hline  &  &  \\ 
		\hline  &  &  \\ 
		\hline  &  &  \\ 
		\hline  &  &  \\ 
		\hline  &  &  \\ 
		\hline  &  &  \\ 
		\hline  &  &  \\ 
		\hline  &  &  \\ 
		\hline  &  &  \\ 
		\hline  &  &  \\ 
		\hline  &  &  \\ 
		\hline  &  &  \\ 
		\hline  &  &  \\ 
		\hline  &  &  \\ 
		\hline  &  &  \\ 
		\hline  &  &  \\ 
		\hline  &  &  \\ 
		\hline  &  &  \\ 
		\hline  &  &  \\ 
		\hline  &  &  \\ 
		\hline 
	\end{tabular} 
	\caption{Functional Requirements}\label{Functional Requirements}
\end{figure}