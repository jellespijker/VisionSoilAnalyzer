The main function of a vision based soil analyzer is clearly evident from its name. The user expects a device which performs an analysis of a soil sample. It does so by capturing and digitizing reflected light of the individual soil particles. This function is illustrated below in an Input-Process-Output\index{Input-Process-Output} (IPO) diagram, see figure \ref{mainIPO}. This is a so called black box approach. It shows an input, an output and a process, where the inner workings are not yet relevant. At this stage it isn't relevant to know what happens in this process.  
\paragraph{Technical system}\index{Technical system}
\begin{sBox}
	Prototype of an intelligent soil microscope
\end{sBox}

\paragraph{Main function}\index{Main function}
\begin{sBox}
	To analyses a dried soil sample, consisting of particle in the range of $ 0.02 [mm] \leq P \leq 2.0 [mm] $ and present a user with information regarding color, texture and structure.
\end{sBox}

\begin{figure}[h]
	\centering
	\begin{tikzpicture}[auto, node distance=5cm, >=latex']
	\node [input, name=input] {};
	\node [block, right of=input] (process) {Process};
	\node [output, right of=process] (output) {};
	
	\draw [draw,->] (input) -- node {Soil sample} (process);
	\draw [->] (process) -- node {Analysis results} (output);
	\end{tikzpicture}
	\caption{Main Input-Process-Output diagram}\index{IPO}\index{Input-Process-Output}\label{mainIPO}
\end{figure}
