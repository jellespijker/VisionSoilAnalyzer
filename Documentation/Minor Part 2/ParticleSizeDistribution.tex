
The normal procedure for creating a Particle Size Distribution uses sieves and weights, to determine the volume of the the particle.

The Sieve mesh size can be perceived as a cross section of a particle, since the particle is only retained in a sieve when it passes through the top sieve but can't pass through the sieve below. The cross section of the particle is at minimum the sieve mesh size of the top sieve, but other dimensions of the particle can exceed the sieve mesh size at which it last passes.
\todo{Work this argument to explain why it's oke to use a 2 dimensional representation of a 3 dimensional particle}