\paragraph{Global workflow}
The soil sample is dried and the user makes sure the particle don’t bond together. A small portion of the sample is placed on a sample plate. Taking care to separate the individual particles as much as possible. The cover is closed and a microscopic camera is positions, in an environment where the light conditions are controlled.

The embedded Linux device takes a snapshot which is analyzed using the following computer algorithms: First the individual soil particles are identified in the image, using various algorithms, such as adaptive contrast stretch, Gaussian blurring, OTSU – optimal thresholds separation. The color information is determined with various matrix calculations, translating the RGB pixel value tot CIE Lab and Redness Index.

The texture information is determined by counting the number of discrete pixels for each individual article. From this the volume is determined. If the scale of each pixel is known, the volume can be given in SI units.

The structure of an individual particle is determined by getting the edge of the pixels. This is done by creating a mask with a morphological erosion algorithm this mask is subtracted of the original image. The contour is translated to a function using the Dijkstra shortest path algorithm. Where each pixel is described as an imaginary complex number representing the radius towards the center of the particle. The vector holding these values are transformed to the frequency space using the Fast Fourier Transformation. The describing complex numbers gained during this transformation are fed into a feedforward Neural Network, which is optimized using Genetic Algorithms and a previously determined learning data set. The output is presented as probability that a certain particle belongs to a predefined category.

The results are presented to the user via a graphical user interface which are show when the device is hooked to a monitor carrying a HDMI input. It’s also possible to present a report in pdf or a native format which can downloaded from the device using a LAN network device or optional Wi-Fi or Bluetooth. Basic human interaction can be performed via an on-board encoder, or optional USB keyboard and/or mouse.