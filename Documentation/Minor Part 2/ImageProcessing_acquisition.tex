A thorough review of the current literature \cite{Spijker14a} identified three properties that can be used in vision based analyzing. These properties are structure \index{structure} (shape), color and texture \index{Texture} (size). When looking closely at sand sample, you notice a multitude of shapes, colors and sizes, each particle is unique and differs from its neighbor. This diversity brings it own challenges. The shape of a particle determines how it will rest on the sample plate. The color and the translucency of the particle, determines how easily it can be segmented or identified from the background. Whilst the size determines the needed focus depth of the microscope. 
\begin{remark}
	In samples, where the particles show a huge spread in size, compared to the mean size, there will be a noticeable difference in focus, between big and small particles. 
\end{remark}

\paragraph{Acquisition strategies}The first prototype is developed in such a way that multiple acquisition strategies\index{acquisition strategies} can be implemented. Each of these tackle different challenges. The quality of the acquired image is the biggest factor in the successful extraction of a particle, but in order to make any valid claim about the sample, a certain amount of particles have to examined. To determine the minimum sample size \index{Minimum sample size}, the following equation can be derived:
\begin{sBox}
	Let the reliability be $95\% \therefore z=1.96$, the probability be $P=50\%$ and the accuracy be $\alpha=5\%$; consider the function:
	\begin{equation}
		z\sqrt{\frac{p\times(1-P)}{n}}\leq\alpha \rightarrow n\geq\frac{-p\times(P-1)\times z^2}{\alpha^2}
	\end{equation}
\end{sBox}
This brings the minimum amount of particles to $384$. With the predefined range of particle sizes ($0.2[mm]\leq P_size \leq 2[mm]$ where $P$ defines a particle) and the limited work area under the microscope, multiple shots have to be taken. Where the sample is rearranged. Between fifteen and twenty shots are usually enough.
\begin{remark}
	The process of rearranging the particles, will be automated in the future. Student of the minor Offshore \& Construction given at the University of Applied Sciences Rotterdam will work on this challenge in the RDM Campus. This minor starts in September 2015. Their product will serve as input for the second prototype iteration. The assignment given to them is described in appendix \ref{RDM_Campus}.
\end{remark}

\paragraph{Acquisition} \index{Acquisition} Each sample is placed in a light condition room, and laid out on a semitransparent white acrylate plate. The sample can be illuminated with a bright field light source, where the light is aimed directly at an object or the particle can be lit with back lighting. See the course notes \cite{ypma_course_2014} for a more in-depth description. The choice for back lighting can be made because translucent particle are harder to segment in a bright field light. The trade off is extra processing time.

After the sample is placed in the light condition room, the microscope takes a image with bright field illumination \index{Bright field illumination} and, if the option is selected, another one with back lighting. \index{Back lighting} Hereafter the sample is rearranged, this is a manual procedure. Once the sample is rearranged a new set of shots is taken. Each image that is acquired from the microscope is defined by a matrix were the values are triples for the RGB \index{RGB} (red, green and blue) values and these are defined by an unsigned byte. 

Each image is stored in a vector using a custom container. This container consists of a bright field image, back light image and a SI-conversion factor \index{SI-conversion factor}. Each time the height is changed, the microscope has to be calibrated so that the relation between pixel and [mm] can be determined. This is done by taking a shot of a disc with known dimensions. A single euro cent can serve for this purpose.

\begin{remark}
	The image is stored in the OpenCV matrix (cv::Mat) container. This container is  designed to handle image processing data and routines. It makes use of memory management and smart pointers to handle the data effectively. 
\end{remark}