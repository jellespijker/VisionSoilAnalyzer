The individual particles have to be identified end segmented from the background. These operations are performed on the enhanced intensity matrix. If the user opted to use back lit and bright field matrices, the enhanced intensity matrices where calculated from the back lit intensity matrices. Otherwise the bright field intensity matrices are used.

\paragraph{Segmentation}\index{Segmentation}
The images are segmented by calculating a threshold value\index{Threshold}. This value is determined by using the Otsu\index{Otsu's method} threshold. \citeauthor{Xu2011956} \cite{Xu2011956} describe that the Otsu threshold is equal to the average of the mean levels of two classes partitioned by this threshold. This threshold value can be iteratively determined.

\begin{sBox}
	Let $\vec{h}$ be a vector of dimension $256$ which represent a count of values in the enhanced intensity matrix $\mat{I}$ with dimensions $m \times n$
	\begin{equation}\label{OtsuMethodEq}
		\frac{1}{t_o}\sum\limits_{i=1}^{t_o} \vec{h}_i = t_o - \frac{1}{256 - t_o}\sum\limits_{i=t_o}^{256} \vec{h}_i
	\end{equation}
\end{sBox}

In order to get more control over the segmentation process, the normal Otsu's method, as shown above is altered. A user now has the option to choose whether bright or dark object are segmented and how much the intensity values may deviation from the mean value. This mean value is either the left or right hand side of the equation \ref{OtsuMethodEq} modified with a scaling factor and the standard deviation, as shown in equation \ref{darkObjEq} and \ref{brightObjEq}.
\begin{sBox}
	Let $ t_o $ be the threshold value obtained with the iteration algorithm used to solve equation \ref{OtsuMethodEq}, $ \alpha $ be the a multiplication factor given by the user and let $\vec{h}$ be a vector of dimension $256$ which represent a count of values in the enhanced intensity matrix $\mat{I}$ with dimensions $m \times n$\\	
	If dark objects are to be obtained
	\begin{equation}\label{darkObjEq}
		t = \frac{1}{t_o}\mu + \frac{1}{2}\alpha\sigma {\rm \ \ where\ \ }\sigma = \sqrt{\frac{1}{t_o} \sum\limits_{i=1}^t (\vec{h}_i - \mu)^2}, {\rm \ \ and\ \ } \mu = \frac{1}{t_o} \sum\limits_{i=1}^t \vec{h}_i
	\end{equation}
	 else
	 \begin{equation}\label{brightObjEq}
	 	t = \frac{1}{t_o}\mu - \frac{1}{2}\alpha\sigma {\rm \ \ where\ \ }\sigma = \sqrt{\frac{1}{256 - t_o} \sum\limits_{i=t}^{256} (\vec{h}_i - \mu)^2}, {\rm \ \ and\ \ } \mu = \frac{1}{256 - t_o} \sum\limits_{i=t}^{256} \vec{h}_i
	 \end{equation} 
\end{sBox}

\paragraph{Binary Image}\index{Binary Image}\index{BW}
The segmented pix
\begin{sBox}
	Let $\mat{B} \subset \mathbb{Z}^n \rightarrow \{0, 1\}$
	\begin{equation}
		0=1		
	\end{equation}
\end{sBox}