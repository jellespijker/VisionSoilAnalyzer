In order to tell something about the individual particles, they first have to be identified and separated from the background. If there are only bright field images in the image vector, the segmentation operations are performed on these otherwise the back lit images are used. 

\paragraph{Segmentation}\index{Segmentation}
The images are segmented by calculating a threshold value\index{Threshold}. This value is determined by using the Otsu\index{Otsu's method} threshold. \citeauthor{Xu2011956} describe that the Otsu threshold is equal to the average of the mean levels of two classes partitioned by this threshold. This threshold can be iteratively determined using the algorithm \ref{threshold_alg}. Let $\mat{H}$ be a histogram representing the count of pixel values in the intensity matrix $\mat{I}$
\begin{algorithm}\label{threshold_alg}
	\caption{Otsu's threshold}
	\begin{algorithmic}[1]
		\Statex
		\Function{GetThresholdLevel}{$\mat{H}$}
		\State $\sigma_r=0$
		\State $\sigma_l=0$
		\State $\bar{l}=0$
		\State $\bar{r}=0$
		\State $T_p=0$
		\State $T_c=\frac{1}{n \times m} \sum\limits_{i=1}^{n \times m} \mat{H}_{i} \times i$
		\While{$T_p != T_c$}
			\State $\sigma=1$
		\EndWhile
	
		\State \Return{$\delta$}
		\EndFunction
	\end{algorithmic}
\end{algorithm}
