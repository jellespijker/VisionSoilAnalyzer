%%%%%%%%%%%%%%%%%%%%%%%%%%%%%%%%%%%%%%%%%
% Memo
% LaTeX Template
% Version 1.0 (30/12/13)
%
% This template has been downloaded from:
% http://www.LaTeXTemplates.com
%
% Original author:
% Rob Oakes (http://www.oak-tree.us) with modifications by:
% Vel (vel@latextemplates.com)
%
% License:
% CC BY-NC-SA 3.0 (http://creativecommons.org/licenses/by-nc-sa/3.0/)
%
%%%%%%%%%%%%%%%%%%%%%%%%%%%%%%%%%%%%%%%%%

\documentclass[a4paper,11pt]{../LatexDocStructures/MTItexMemo} % Set the paper size (letterpaper, a4paper, etc) and font size (10pt, 11pt or 12pt)

\usepackage{parskip} % Adds spacing between paragraphs
\setlength{\parindent}{15pt} % Indent paragraphs

%----------------------------------------------------------------------------------------
%	MEMO INFORMATION
%----------------------------------------------------------------------------------------

\memoto{Johan Van Eijden; Richard Kaandorp} % Recipient(s)

\memocc{Hans Bink; Rick Lotman; Mario Alvarez Grima; Wouter Bron; Jeroen van Elburg}

\memofrom{Jelle Spijker} % Sender(s)

\memosubject{Thesis assignment vision based soil analyzer} % Memo subject

\memodate{\today} % Date, set to \today for automatically printing todays date

\logo{\includegraphics[width=0.3\textwidth]{../Pictures/MTI_logo.png}} % Institution logo at the top right of the memo, comment out this line for no logo

%----------------------------------------------------------------------------------------

\begin{document}

\maketitle % Print the memo header information

%----------------------------------------------------------------------------------------
%	MEMO CONTENT
%----------------------------------------------------------------------------------------
\paragraph{Background}\mbox{} \\
During the minor embedded vision design taught at the HAN university of applied sciences in Arnhem, a prototype for a vision based soil analyzer was developed. This device analyzes soil with regard to color, structure and texture. It does so by applying a range of algorithms to a digital obtained image. Processing is done on a microcontroller. Because of the novelty of this first prototype IHC MTI B.V. has indicated that they are willing to further explorer the possibilities of a vision based soil analyzer.

\paragraph{Assignment}\mbox{} \\
This project lends itself for a bachelor degree thesis mechanical engineering, where development of a vision based soil analyzer has a central focus. This project will be executed under the auspices of MTI Holland. The expected end-result will be a \textbf{technical dossier} which contains a design and backs its choices with appropriate documents (e.g market research, literature review, product requirement specifications, functional design, construction drawings, material selection etc.). The project follows a methodical approach consisting of a preliminary research, problem definition, determining procedures and methods followed by an actual design and verification of this design. This approach is also know as the Kroonenberg method.

The project has a duration of 26 weeks and will commence at the end of November. The student will work three days a week at IHC MTI located at Delftechpark in Delft.

\paragraph{Company profile}\mbox{} \\
IHC MTI B.V. is based in Delft and is the global leading knowledge center in the area of translating knowledge of dredging, mining and deep-sea mining processes into the specification, design and application of equipment. 

MTI currently employs approximately 60 dedicated staff. These staff members include experienced engineers in the field of dredging engineering, mining engineering, process engineering, geophysics, geology and geo-techniques, ship dynamics and design, mechanical engineering, fluid dynamics engineering and multi-phase dynamics engineering. About 15\% of our staff hold a PhD degree.

%----------------------------------------------------------------------------------------

\end{document}