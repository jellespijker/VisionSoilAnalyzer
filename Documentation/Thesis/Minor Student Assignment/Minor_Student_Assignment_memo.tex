%%%%%%%%%%%%%%%%%%%%%%%%%%%%%%%%%%%%%%%%%
% Memo
% LaTeX Template
% Version 1.0 (30/12/13)
%
% This template has been downloaded from:
% http://www.LaTeXTemplates.com
%
% Original author:
% Rob Oakes (http://www.oak-tree.us) with modifications by:
% Vel (vel@latextemplates.com)
%
% License:
% CC BY-NC-SA 3.0 (http://creativecommons.org/licenses/by-nc-sa/3.0/)
%
%%%%%%%%%%%%%%%%%%%%%%%%%%%%%%%%%%%%%%%%%

\documentclass[a4paper,11pt]{../LatexDocStructures/MTItexMemo} % Set the paper size (letterpaper, a4paper, etc) and font size (10pt, 11pt or 12pt)

\usepackage{parskip} % Adds spacing between paragraphs
\setlength{\parindent}{15pt} % Indent paragraphs

\renewcommand{\deg}{\ensuremath{^{\circ}}\xspace}
%----------------------------------------------------------------------------------------
%	MEMO INFORMATION
%----------------------------------------------------------------------------------------

\memoto{Jeroen van Elburg; Richard Kaandorp} % Recipient(s)

\memocc{Mario Alvarez Grima; Wouter Bron; Jort van Wijk}

\memofrom{Jelle Spijker} % Sender(s)

\memosubject{Conceptueel ontwerp van een draagbare gronddroger } % Memo subject

\memodate{\today} % Date, set to \today for automatically printing todays date

\logo{\includegraphics[width=0.3\textwidth]{../Pictures/MTI_logo.png}} % Institution logo at the top right of the memo, comment out this line for no logo

%----------------------------------------------------------------------------------------

\begin{document}

\maketitle % Print the memo header information

%----------------------------------------------------------------------------------------
%	MEMO CONTENT
%----------------------------------------------------------------------------------------
\paragraph{Achtergrond}\mbox{} \\
Tijdens de minor embedded vision design (EVD) is een grond analyseer apparaat ontwikkeld. Deze analyseert gedroogde grondmonsters met behulp van een digitale camera en doet een uitspraak over kleur, structuur en textuur. Deze vision based soil analyzer (VSA) onderscheid zichzelf doordat hij handzaam en draagbaar is. hierdoor kan op locatie een grondmonster geanalyseerd worden. De meeste apparaten welke grond analyseren zijn log, groot, zwaar en staan in laboratoria.

Het product wordt verder ontwikkeld tijdens een werktuigbouwkundige afstudeeropdracht in opdracht van IHC MTI B.V.. Een van de ge\"identificeerde deel-problemen van deze opdracht is het drogen van een grondmonster op locatie. Grond bestaat uit korrels welke vari\"eren in grote van $ 2 [\mu m] $ tot $ 2 [m] $ en bestaan uit organische en anorganische deeltjes. Volgens het huidige proced\'e wordt een dergelijk grondmonster gedurende een periode van $ 24 [hr] $ bij $ 80 [ \deg C] $ gedroogd in een oven. Deze methode leent zich niet voor veld analyses. Het is daarom van belang om een methode of apparaat te ontwikkelen welke op locatie snel een grondmonster kan drogen.

\paragraph{Opdracht}\mbox{} \\
Ontwikkel een concept oplossing waarmee het vochtgehalte van een grondmonster op locatie met een maximaal gewicht van $ 50 [gr] $ en bestaat uit korrels vari\"erend in grote tussen $ 200 [\mu m] $ tot $ 2 [mm] $, binnen een periode van $ 10 [min] $, verlaagt kan worden tot een vochtgehalte welke gelijk is aan de omgevingslucht.

\paragraph{Bedrijfsprofiel}\mbox{} \\
IHC MTI B.V. heeft als basis het Delftechpark in Delft. Zij zijn wereldwijd het kenniscentrum voor het ontwikkelen van bagger, mijnbouw en diepzee mijnbouw specificaties, ontwerpen en ontwikkelingen van apparaten. MTI is een dochter onderneming van de Koninklijke IHC een wereldleider in de bouw van bagger- en marine \& offshore schepen.

MTI heeft ongeveer 60 man in dienst waaronder ervaren ingenieurs in de disciplines, dredging engineering, mining engineering, process engineering, mechanical engineering, fluid dynamic engineering, geophysics, geology, ship dynamics and design and multi-phase dynamics. 15\% bezit een doctorsgraad.

%----------------------------------------------------------------------------------------

\end{document}